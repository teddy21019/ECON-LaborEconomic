\documentclass{article}
\usepackage{amsmath}

\begin{document}
\section{The Simple Roy Model}
We consider a selection problem. 
        Assume for any of the students in NTU, student $i$, the wage of going to a small company is $w_{0i}$, 
        and the wage of going to a big company is $w_{1i}$. 

        The wage is modeled to be 
        \begin{align}
            w_{0i} &= \mu_0 + \epsilon_{0i}\\
            w_{1i} &= \mu_1 + \epsilon_{1i}
        \end{align}
        We assume that 
        $\epsilon_{0i} \sim \mathcal{N}(0,\sigma_0^2), \epsilon_{1i} \sim \mathcal{N}(0,\sigma_1^2), Cov(\epsilon_{0i}, \epsilon_{1i}) = \sigma_{01}$.
        
        Let $D=1$ be the student choosing to go to a big company. 
        The student chooses to go to a big company if it offers a higher wage, so 
        \begin{equation*}
            D = 1\{w_{1i} > w_{0i}\} = 1\{\mu_1 + \epsilon_{1i} > \mu_0 + \epsilon_{0i}\}
        \end{equation*}

        We now observe the average wage of a student that goes to a big company, that is, $E[w_{1i} \mid D=1]$. 
        As a economist, we are interested in knowing \emph{What will be the expected wage if this student didn't go to a big company, but went to a small company instead?}
        , in other words, what is $E[w_{0i} \mid D=1]$? 

        This is never observed (a.k.a counterfactual), because we can't reverse time and tell the student to change its mind and go to a small company. 
        However, we can somehow derive this unobserved potential outcome by using our knowledge in statistics.

    \begin{enumerate}
        \item Given $X \sim \mathcal{N}(0, 1)$, proof that 
        \begin{equation*}
            f(x \mid x > a) = \frac{\phi(x)}{1 - \Phi(a)}
        \end{equation*}
        where $\phi(\cdot)$ is the PDF of a normal distribution, and $\Phi(\cdot)$ is the CDF. 
        This is the PDF of a \emph{truncated normal distribution}
        
        \item Find the expectation value $E[X \mid X>a]$.
        \item Following 2. Now $Y \sim \mathcal{N}(0,\sigma_y^2)$, and $Corr(X,Y) = \rho_{xy}$.
        
        Use the result
        \begin{equation*}
            E[Y \mid X > a] = \frac{1}{\Pr(X>a)} \int_a^\infty E[Y \mid X=x] f_X(x) dx 
        \end{equation*},
        prove that 
        \begin{equation*}
            E[Y \mid X>a] = \rho_{xy} \sigma_y E[X \mid X>a]
        \end{equation*}
        \item Now, consider the estimation we are interested in, $E[w_{0i} \mid D=1]$. 
        We rewrite it as 
        \begin{align*}
            E[w_{0i} \mid D=1] &= E[w_{0i} \mid w_{1i} > w_{0i}] \\
            &= E[w_{0i} \mid \mu_1 + \epsilon_{1i} > \mu_0 + \epsilon_{0i}] \\
            &= \mu_0 + E[\epsilon_{0i} \mid \epsilon_{1i} - \epsilon_{0i} > \mu_0 - \mu_1]
        \end{align*}
        
        Define $\nu_i = \epsilon_{1i} - \epsilon_{0i}$, find $\sigma_v^2 = Var(\nu_i)$.
        \item Prove that 
        \begin{equation*}
            E[w_{0i} \mid D=1] = \mu_0 + \rho_{o,\nu}\sigma_0 E\left[\frac{\nu_i}{\sigma_\nu} \mid \frac{\nu_i}{\sigma_\nu} > \frac{\mu_0 - \mu_1}{\sigma_\nu}\right]
        \end{equation*}
        Where $\rho_{0,\nu}$ is the correlation coefficient of $\epsilon_{0i}$ and $\nu_i$. 
        
        \emph{Hint. What is $\epsilon_{0i} \mid \nu_i \sim ?$}
        \item Finally, prove that 
        \begin{equation*}
        E[w_{0i} \mid D=1] = \mu_0 + \frac{\sigma_0 \sigma_1}{\sigma_\nu}\left(\rho_{01} - \frac{\sigma_0}{\sigma_1}\right)\frac{\phi(z)}{1-\Phi(z)}
        \end{equation*}
        , where $z \equiv \frac{\mu_0 - \mu_1}{\sigma_v}$
    \end{enumerate}

\end{document}