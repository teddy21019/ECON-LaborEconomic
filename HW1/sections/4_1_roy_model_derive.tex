\subsection{Derivation of Roy Model}

We first start from considering the probability of migration. 
One migrates under the condition

\begin{equation}
    w_{1i} > w_{0i} + C
\end{equation}

Where $w_{0i} = \mu_{0i} + \epsilon_0$ and $w_{1i} = \mu_{1i} + \epsilon_1$, 
and the error terms together follow a joint-normal distribution

\begin{equation}
    \label{eq:def_w}
    \mqty(\epsilon_0 \\ \epsilon_1) 
    \sim
    \mathcal{N} \qty(
        \mqty(0 \\ 0),
        \mqty(\sigma_0^2, \sigma_{01} \\
              \sigma_{01}, \sigma_1^2)
    )
\end{equation}

The probability of migration is then, according to \refeq{eq:def_w}

\begin{align}
    P(w_{1i} > w_{0i} + C) &= P(v_i > \mu_0 - \mu_1 + C) \nonumber \\
    &=1 - \Phi(\frac{\mu_0 - \mu_1 + C}{\sigma_\nu}) \nonumber \\
    &=1 - \Phi(z) \nonumber
\end{align}

The expected wage of an imigrant is then

\newcommand{\given}{\mid}

\emph{Note: Subscripts i are neglected for simplicity}
\begin{align*}
    \E(w_{0} \given I) =& \E(w_{0} \given v > \mu_0 - \mu_1 + C)  \\
    =& \mu_0 + \E(\epsilon_0 \given  v > \mu_0 - \mu_1 + C )  \\
    =& \mu_0 + \E \left(
            \epsilon_0 \given \frac{\nu}{\sigma_\nu} > \frac{\mu_0 - \mu_1 + C}{\sigma_\nu}
            \right)
\end{align*}

A linear combination of random variables following normal distribution
also follows a normal distribution. Therefore $\E\left(\epsilon_0 \given \frac{\nu}{\sigma_\nu}\right)$
is just a conditional expectation of a bivariate normal distribution
consisting $\epsilon$ and $\nu/\sigma_\nu$.

Since for a bivariate normal distribution, we have 
\begin{equation}
    \label{eq:biv-exp}
    \E[X \given Y=y] = \mu_x + \rho_{xy}\frac{\sigma_x}{\sigma_y}(y - \mu_y) 
\end{equation}

Substituding $X=\epsilon_0$ and $Y=\nu$ into \refeq{eq:biv-exp}, we get

\begin{equation}
    \label{eq:biv_normal_exp_0}
    \E(\epsilon_0 \given \nu)=\rho_{0\nu}\frac{\sigma_0}{\sigma_\nu}\nu
\end{equation}

and hence

\begin{equation*}
    \E(\epsilon_0 \given \frac{\nu}{\sigma_\nu})=
    \left(\rho_{0\nu} \frac{1}{\sigma_\nu}\right)
    \frac{\sigma_0}{\sigma_\nu \frac{1}{\sigma_\nu^2}}
    \frac{\nu}{\sigma_\nu}=\frac{\rho_{0\nu}\sigma_0 }{\sigma_\nu}\nu
\end{equation*}

Therefore
\begin{align}
    \E(w_{0} \given I) =& 
     \mu_0 + \E \left(
            \epsilon_0 \given \frac{\nu}{\sigma_\nu} > \frac{\mu_0 - \mu_1 + C}{\sigma_\nu}
            \right) \nonumber \\
    &= \mu_0 + \rho_{0\nu}\sigma_0 \E \left(
        \frac{\nu}{\sigma_\nu} 
        \given
        \frac{\nu}{\sigma_\nu} > \frac{\mu_0 - \mu_1 + C}{\sigma_\nu}
    \right) \label{eq:before_trunc}
\end{align}

Realize that the last expectation \refeq{eq:before_trunc}
is the expected value of a truncated normal distribution, 
which can rewrite it as 

\begin{equation}
    \label{eq:exp_wo_1}
    \E(w_{0} \given I) = 
    \mu_0 + \rho_{0\nu}\sigma_0 \frac{\phi(z)}{1-\Phi(z)}
\end{equation}

Since the correlation coefficient can be negative,
 we know nothing about this result
Let us preceed and solve for $\rho_{0\nu}$ 
further for better insight. 


\begin{equation*}
    \rho_{0v} = \frac{\sigma_{0\nu}}{\sigma_0\sigma_\nu}
\end{equation*}
\begin{equation}
    \label{eq:cov_0_nu}
    \sigma_{0\nu}=cov(\epsilon_0, \nu)=\E(\epsilon_0 (\epsilon_1 - \epsilon_0)) = \sigma_{01}-\sigma_0^2
\end{equation}

Substitude the result from Eq.~\refeq{eq:cov_0_nu} and $\rho_{01}=\sigma_{01}/\sigma_0\sigma_1$, 
\refeq{eq:exp_wo_1} now becomes
\begin{equation}
    \label{eq:E_0_I}
    \E(w_{0} \given I) = 
    \mu_0 + \frac{\sigma_{01}-\sigma_0^2}{\sigma_\nu} \frac{\phi(z)}{1-\Phi(z)}
    =\mu_0 + \frac{\sigma_0 \sigma_1}{\sigma_\nu}\left(
        \rho - \frac{\sigma_0}{\sigma_1}
    \right)\frac{\phi(z)}{1-\Phi(z)}
\end{equation}

Similarily, for $\E(w_1 \given I)$, we have

\begin{align*}
    \E(w_{1} \given I) =& \E(w_{1} \given v > \mu_0 - \mu_1 + C)  \\
    =& \mu_1 + \E(\epsilon_1 \given  v > \mu_0 - \mu_1 + C )  \\
    =& \mu_1 + \E \left(
            \epsilon_1 \given \frac{\nu}{\sigma_\nu} > \frac{\mu_0 - \mu_1 + C}{\sigma_\nu}
            \right)
\end{align*}

Substituding 1 for 0 in \refeq{eq:biv_normal_exp_0}, we get
\begin{equation}
    \label{eq:biv_normal_exp_1}
    \E(\epsilon_1 \given \frac{\nu}{\sigma_\nu})
    =\frac{\rho_{1\nu}\sigma_1}{\sigma_\nu}\nu
\end{equation}

\refeq{eq:cov_0_nu} now becomes
\begin{equation}
    \label{eq:cov_1_nu}
    \sigma_{1\nu}=cov(\epsilon_, \nu)=
    \E(\epsilon_1 (\epsilon_1 - \epsilon_0)) = \sigma_{1}^2-\sigma_{01}
\end{equation}

Hence together we get
\begin{equation}
    \label{eq:E_1_I}
    \E(w_{1} \given I) = 
    \mu_1 + \rho_{1\nu}\sigma_1 \frac{\phi(z)}{1-\Phi(z)}
    =\mu_1 + \frac{\sigma_0 \sigma_1}{\sigma_\nu}\left(
        \frac{\sigma_1}{\sigma_0} - \rho
    \right)\frac{\phi(z)}{1-\Phi(z)}
\end{equation}



\subsection{$Q_0>0$ and $Q_1<0$ ?}

Accordind to \refeq{eq:E_0_I} and \refeq{eq:E_1_I}, this would imply that
$$
\begin{cases}
    \rho>\frac{\sigma_0}{\sigma_1}\\
    \rho>\frac{\sigma_0}{\sigma_1}
\end{cases}
$$
But either will exceed 1, making it mathematically imposible to happen since correlation coefficient must be bounded under 1.

Intuitively, it means that people migrate to a place where expectation wage is lower than that in local.
which is weird and irrational.