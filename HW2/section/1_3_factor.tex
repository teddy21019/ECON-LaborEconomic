\subsection{Identification of Factor Model}

Labeling the equations,

\begin{subequations}
    \begin{align}
        y_{i,t} &= \nu_{i,t} + \epsilon_{i,t} \label{eq:y-it} \\
        \nu_{i,t} &= \rho \nu_{i,t-1} + \xi_{i,t} \label{eq:nu-it}
    \end{align}
\end{subequations}

\subsubsection{$\rho$}

Substituting \refeq{eq:nu-it} into \refeq{eq:y-it}, we get 

\begin{equation}
    y_{i,t} = \rho \nu_{i,t-1} + \xi_{i,t}  + \epsilon_{i,t} \label{eq:y_t-1_nu}
\end{equation}

By \refeq{eq:y-it}, we know $y_{i,t-1} = \nu_{i,t-1} + \epsilon_{i,t-1}$, hence by \refeq{eq:y_t-1_nu} we get 
\begin{equation}
    y_{i,t} = \rho y_{i,t-1} - \rho \epsilon_{i,t-1} + \epsilon_{i,t} + \xi_{i,t}
\end{equation}

The linear relation between $y_{i,t}$ and $y_{i,t-1}$ guarantees that there cannot exist more than one $\rho$s. It is then obvious that $\rho$ is identified.

\subsubsection{$\sigma_\epsilon^2$}

\subsubsection{$\sigma_\xi^2$}

\subsubsection{Estimator}