\subsection{Propensity Score Simulation}

\subsubsection{Code for Simulation}

\fbox{Note that this section covers all the subquestions in this problem.}

\subsubsection{Example of $X_1$}

$X_1$ is the common factor for both outcomes which has the same marginal effect. In the example of migration, think of $X_1$ as the experience, which in both countries increases its wage in the same scale.

$\beta_1$ does not affect the choice of outcomes, but it equally affects both $w_0$ and $w_1$, therefore it can be identified by looking at the value of $w$. 


\subsubsection{Define the Propensity Score}

\begin{subequations}
    \begin{align}
        w_0 &= \mu_0 + \beta_1 X_1  + \epsilon_0 \label{eq:w0} \\
        w_1 &= \mu_1 + \beta_1 X_1 + \beta_2 X_2 + \epsilon_1 \label{eq:w1}
    \end{align}
\end{subequations}

The propensity score is defined as

\begin{equation}
    \ps(x_1, x_2) = \Pr(\D = 1 \given X_1=x_1, X_2=x_2) \label{eq:roy_pscore}
\end{equation}


\subsubsection{Derive the Propensity Score}
Note that in Roy model, people choose $\D = 1$ if $w_1 > w_0 + C$, therefore

\begin{align*}
    \ps(x_1, x_2) &= \Pr(w_1 > w_0 + C) \\
    &= \Pr(\mu_1 + \beta_1 x_1 + \beta_2 x_2 + \epsilon_1 > \mu_0 + \beta_1 x_1  + \epsilon_0 + C) \\
    &= \Pr(\epsilon_0 - \epsilon_1 < 
        \mu_1 - \mu_0 + \beta_2 x_2 - C
    ) \\
    &= \Pr\left(
        \frac{\nu}{\sigma_\nu} < 
        \frac{\mu_1 - \mu_0 + \beta_2 x_2 - C}{\sigma_\nu}
        \right)
\end{align*}

Where $\nu \equiv \epsilon_0 - \epsilon_1$

% \subsubsection{}