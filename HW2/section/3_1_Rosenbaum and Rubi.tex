\subsection{Proof of Rosenbaum and Rubin}

Define the propensity score 

\newcommand{\ps}{\mathscr{P}}
\newcommand{\prob}{\mathrm{Pr}}
\newcommand{\D}{\mathcal{D}}

\begin{equation}
    \label{eq:pscore}
    \ps(x) \equiv \prob(\D = 1 \given X=x)
\end{equation}
Given the conditional independence assumption (CIA)

\begin{equation} 
    \label{eq:CIA}
    \{ y_0, y_1 \}\indep x
\end{equation}

We want to show that 
\begin{equation}
    \label{eq:RR}
    \{ y_0, y_1 \} \indep D \given \ps(x)
\end{equation}

Intuitively, the propensity is a mapping from the space of $X$ to $[0,1]$, such that under this propensity, the potential outcome is independent of its choice $d$.

\begin{proof}
    Consider $\Pr(\D = 1 \given y_0, y_1, \ps(x))$. By the law of iterated expectation, it is equivalent to

    $$
        \E\left[
            \Pr(
                \D = 1 \given y_0, y_1, \ps(x), x
            )
            \given
            y_0, y_1, \ps(x)
        \right]
    $$
    Since $\ps(x)$ is a function of x, we can neglect $\ps$
    $$
        =\E\left[
            \Pr(
                \D = 1 \given y_0, y_1, x
            )
            \given
            y_0, y_1, \ps(x)
        \right]
    $$
    We assume CIA, as stated in \refeq{eq:CIA}, hence given $x$, the potential outcome is independent to the choice
    $$
    =\E\left[
        \Pr(
            \D = 1 \given x
        )
        \given
        y_0, y_1, \ps(x)
    \right]
    $$
    Notice that by definition in \refeq{eq:pscore}, we can write
    $$
    =\E\left[
            \ps(x)
        \given
        y_0, y_1, \ps(x)
    \right]
    =\ps(x)
    $$
    that gives
    $$
    \Pr(\D = 1 \given y_0, y_1, \ps(x))
    = \ps(x)
    $$

    We have shown that $\D$ given the propensity score is completely independent of the potential outcomes, therefore prooving \refeq{eq:RR}.


\end{proof}