\subsection{ATE}

The ATE measures

\begin{equation*}
    \E[Y_1 - Y_0]
\end{equation*}

It is the expected difference of future average earning on whether one attends college or not.

\subsection{ATT}

The ATT measures 

\begin{equation*}
    \E[Y_1 - Y_0 \given D=1]
\end{equation*}

It is the expected difference of future average earning for people that attended college. 
This is a what if question: \emph{What will be a bachelors' future earning if he doesn't go to college?}

\subsection{PRTE}

Although the treatment effect mentioned above tells us some aspects of potential outcome, 
it is not really useful for policy makers. For example, given a ATE, a consultant tell the policy makers 
\emph{This is the average different of earning, but I can't tell you whether changing the tuition makes any different.}
Similarly, the ATT tell the policy maker \emph{I don't know whether this policy make people tend to attend college more or not, 
but if they already attend college, this will be how much they earn.}

The policy relevant treatment effect, on the other hand, 
considers together the difference in the outcome we are interested in ($Y$) as well as the 
change in the decision making process according to a policy driven channel ($Z*$ changing $D$). 

With PRTE, we can estimate the effect of conducting a policy. 

\subsection{Relationship with LATE}

The local average treatment effect is defined as

\begin{equation*}
    \operatorname{LATE} = \E[Y_1 - Y-0 \given D_z = z, D_{z*} = z*]
\end{equation*}

and its estimation
\begin{equation*}
    \widehat{LATE} = \frac
    {\E[Y \given Z=z*] - \E[Y \given Z=z]}
    {\E[D \given Z=z*] - \E[D \given Z=z]}
\end{equation*}

Notice that the only different with PRTE is the condition. 
Requiring $\E(Y*) = \E[Y \given Z=z*]$ and vice versa is the key for the two to match.
Intuitively, the PRTE will be equivalent to LATE if the samples that we consider 
are the ones that given a change in policy, it will definitely react to it. 